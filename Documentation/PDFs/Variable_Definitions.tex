\documentclass[11pt,letterpaper]{article}

% Packages
\usepackage[utf8]{inputenc}
\usepackage[margin=1in]{geometry}
\usepackage{graphicx}
\usepackage{hyperref}
\usepackage{booktabs}
\usepackage{longtable}
\usepackage{fancyhdr}
\usepackage{xcolor}
\usepackage{listings}

% Setup
\hypersetup{
    colorlinks=true,
    linkcolor=blue,
    filecolor=magenta,      
    urlcolor=cyan,
    pdftitle={Variable Definitions - Failing Banks},
}

\pagestyle{fancy}
\fancyhf{}
\rhead{Failing Banks Replication}
\lhead{Variable Definitions}
\rfoot{Page \thepage}

% Title
\title{\textbf{Failing Banks R Replication}\\[0.5em]\Large Variable Definitions}
\author{Data Dictionary}
\date{October 22, 2025}

\begin{document}

\maketitle

\tableofcontents
\newpage

\section{Overview}

This document provides comprehensive definitions for all 70+ variables used in the Failing Banks R replication package. Variables are organized by category for easy reference.

\subsection{Naming Conventions}

\begin{itemize}
    \item \textbf{F[N]\_failure}: Binary indicator for failure within N years/quarters
    \item \textbf{L\_variable}: Lagged value (one period)
    \item \textbf{L[N]\_variable}: Lagged value (N periods)
    \item \textbf{\_ratio}: Expressed as proportion (0-1 scale)
    \item \textbf{\_growth}: Log difference (approximates percentage change)
\end{itemize}

\section{Identification Variables}

\begin{longtable}{p{3cm}p{2cm}p{9cm}}
\caption{Bank and Time Identifiers}\label{tab:identifiers}\\
\toprule
\textbf{Variable} & \textbf{Type} & \textbf{Definition} \\
\midrule
\endfirsthead
\multicolumn{3}{c}{\textit{(continued from previous page)}}\\
\toprule
\textbf{Variable} & \textbf{Type} & \textbf{Definition} \\
\midrule
\endhead
\midrule
\multicolumn{3}{r}{\textit{(continued on next page)}}\\
\endfoot
\bottomrule
\endlastfoot

bank\_id & Integer & Unique bank identifier assigned by data source. Range: 1-99999. Consistent across time periods. \\
year & Integer & Calendar year from report date. Range: 1863-2024. Annual for historical era. \\
quarter & Integer & Calendar quarter (1-4). Only available for modern era (1959+). \\
quarter\_number & Integer & Alias for quarter variable. Same range and definition. \\
report\_date & Date & Date of call report submission. Date object. Range: 1863-2024. \\
era\_group & Categorical & Era classification: "Historical" (1863-1941) or "Modern" (1959-2024). \\
age & Integer & Bank age in years since charter. Range: 0-150 years. \\
\end{longtable}

\section{Failure Variables}

\subsection{Core Failure Indicators}

\begin{longtable}{p{3cm}p{2cm}p{9cm}}
\caption{Primary Failure Variables}\label{tab:failure-core}\\
\toprule
\textbf{Variable} & \textbf{Type} & \textbf{Definition} \\
\midrule
\endfirsthead
\multicolumn{3}{c}{\textit{(continued from previous page)}}\\
\toprule
\textbf{Variable} & \textbf{Type} & \textbf{Definition} \\
\midrule
\endhead
\midrule
\multicolumn{3}{r}{\textit{(continued on next page)}}\\
\endfoot
\bottomrule
\endlastfoot

failed\_bank & Binary & Permanent indicator: equals 1 if bank ultimately failed, 0 otherwise. \\
days\_to\_failure & Integer & Days until failure (negative before failure). Range: -10000 to 0. NA if bank never fails. \\
months\_to\_failure & Integer & Months until failure. Calculated as days\_to\_failure / 30.4375. Range: -300 to 0. \\
quarters\_to\_failure & Integer & \textbf{KEY VARIABLE}. Quarters until failure. Calculated as days\_to\_failure / 91.25. Range: -300 to 0. Negative values indicate bank has not yet failed. \\
time\_to\_fail & Integer & Years until failure. Calculated as floor(days\_to\_failure / 365). Range: -20 to 0. \\
receivership\_date & Date & Date receiver appointed (historical era only, 1863-1937). From OCC data. \\
fail\_day & Date & Failure date for modern era banks (1959-2024). From FDIC data. \\
final\_year & Integer & Last year bank appears in dataset. May not be failure year if bank merged/exited. \\
\end{longtable}

\subsection{Horizon-Specific Failure Indicators}

These variables indicate failure within specific time horizons. Created in analysis scripts (32, 35).

\begin{longtable}{p{3cm}p{2cm}p{9cm}}
\caption{Failure Horizon Indicators}\label{tab:failure-horizons}\\
\toprule
\textbf{Variable} & \textbf{Values} & \textbf{Definition} \\
\midrule
\endfirsthead
\multicolumn{3}{c}{\textit{(continued from previous page)}}\\
\toprule
\textbf{Variable} & \textbf{Values} & \textbf{Definition} \\
\midrule
\endhead
\midrule
\multicolumn{3}{r}{\textit{(continued on next page)}}\\
\endfoot
\bottomrule
\endlastfoot

F1\_failure & 0 or 100 & Fails within 1 year (4 quarters). Formula: $100 \times (\text{quarters\_to\_failure} \geq -4 \text{ and } \leq -1)$ \\
F2\_failure & 0 or 100 & Fails within 2 years (8 quarters). Used in cross-section analysis (Script 32). \\
F3\_failure & 0 or 100 & \textbf{PRIMARY OUTCOME}. Fails within 3 years (12 quarters). Formula: $100 \times (\text{quarters\_to\_failure} \geq -12 \text{ and } \leq -1)$ \\
F4\_failure & 0 or 100 & Fails within 4 years (16 quarters). Cross-section analysis. \\
F5\_failure & 0 or 100 & Fails within 5 years (20 quarters). Alternative horizon in Script 35. \\
F6\_failure & 0 or 100 & Fails within 6 years (24 quarters). Cross-section only. \\
F1\_failure\_run & 0 or 1 & Fails within 1 year \textit{AND} experienced bank run. Conditional indicator. \\
F3\_failure\_run & 0 or 1 & Fails within 3 years \textit{AND} experienced bank run. Conditional indicator. \\
\end{longtable}

\section{Balance Sheet Variables}

\begin{longtable}{p{3cm}p{2cm}p{9cm}}
\caption{Core Balance Sheet Items}\label{tab:balance-sheet}\\
\toprule
\textbf{Variable} & \textbf{Type} & \textbf{Definition} \\
\midrule
\endfirsthead
\multicolumn{3}{c}{\textit{(continued from previous page)}}\\
\toprule
\textbf{Variable} & \textbf{Type} & \textbf{Definition} \\
\midrule
\endhead
\midrule
\multicolumn{3}{r}{\textit{(continued on next page)}}\\
\endfoot
\bottomrule
\endlastfoot

assets & Continuous & Total assets in nominal dollars. Range: \$100 - \$1 trillion. Sum of all asset categories from call reports. \\
deposits & Continuous & Total deposits in nominal dollars. Range: \$50 - \$1 trillion. Includes demand, time, and savings deposits. \\
loans & Continuous & Total loans in nominal dollars. Range: \$20 - \$100 billion. Sum of all loan categories. \\
liquid & Continuous & Liquid assets in nominal dollars. Range: \$10 - \$100 billion. \textit{Historical}: cash + securities + due from banks. \textit{Modern}: cash + securities + federal funds purchased. \\
equity & Continuous & Total equity/book value in nominal dollars. Range: \$5 - \$10 billion. Formula: capital + surplus + undivided\_profits. \\
capital & Continuous & Capital stock in nominal dollars. Par value of outstanding shares. Primary equity component for historical banks. \\
surplus & Continuous & Surplus/retained earnings in nominal dollars. Accumulated retained earnings over time. Historical era. \\
undivided\_profits & Continuous & Current period undivided profits. Historical era. \\
log\_assets & Continuous & Natural log of total assets. Range: 5-20. Used as bank size measure in regressions. \\
\end{longtable}

\section{Financial Ratios}

\subsection{Key Risk Measures}

\begin{longtable}{p{3.5cm}p{8.5cm}}
\caption{Primary Financial Ratios}\label{tab:ratios}\\
\toprule
\textbf{Variable} & \textbf{Definition \& Interpretation} \\
\midrule
\endfirsthead
\multicolumn{2}{c}{\textit{(continued from previous page)}}\\
\toprule
\textbf{Variable} & \textbf{Definition \& Interpretation} \\
\midrule
\endhead
\midrule
\multicolumn{2}{r}{\textit{(continued on next page)}}\\
\endfoot
\bottomrule
\endlastfoot

equity\_ratio & \textbf{SOLVENCY MEASURE}. equity / assets. Range: 0.01-0.50 (1\%-50\%). Higher values indicate better capitalization and lower insolvency risk. \\
loan\_ratio & \textbf{ASSET RISK}. loans / assets. Range: 0.20-0.90 (20\%-90\%). Higher values indicate more lending activity but potentially higher credit risk. \\
liquid\_ratio & \textbf{LIQUIDITY MEASURE}. liquid / assets. Range: 0.05-0.60 (5\%-60\%). Higher values indicate better ability to meet short-term obligations. \\
surplus\_ratio & Profitability indicator (historical). surplus / equity. Range: 0-2. Higher values suggest accumulated profitability. \\
cash\_ratio & Narrow liquidity measure (historical). cash\_reserves / assets. Range: 0-0.50. Strictest liquidity measure. \\
income\_ratio & Profitability measure (modern only, post-1941). net\_income / assets. Range: 0-0.20 (0\%-20\% ROA). \\
\end{longtable}

\section{Growth Variables}

\begin{longtable}{p{3.5cm}p{8.5cm}}
\caption{Growth Measures}\label{tab:growth}\\
\toprule
\textbf{Variable} & \textbf{Definition \& Calculation} \\
\midrule
\endfirsthead
\multicolumn{2}{c}{\textit{(continued from previous page)}}\\
\toprule
\textbf{Variable} & \textbf{Definition \& Calculation} \\
\midrule
\endhead
\midrule
\multicolumn{2}{r}{\textit{(continued on next page)}}\\
\endfoot
\bottomrule
\endlastfoot

assets\_growth & Annual asset growth rate. Formula: $\ln(\text{assets}_t) - \ln(\text{assets}_{t-1})$. Range: -0.50 to +0.50 (-50\% to +50\%). Year-over-year growth. \\
deposits\_growth & Annual deposit growth rate. Same formula as assets\_growth but for deposits. Range: -0.50 to +0.50. \\
growth & \textbf{PRIMARY GROWTH MEASURE}. 3-year asset growth. Formula: $\ln(\text{assets}_t) - \ln(\text{assets}_{t-3})$. Range: -1 to +1. Used for growth quintile analysis. \\
growth\_cat & Growth quintile (1-5). Within-year quintiles of growth variable. 1 = slowest growing (Q1), 5 = fastest growing (Q5). Created in Scripts 32 and 35. \\
L3\_assets & Assets lagged 3 periods. Used to calculate growth variable. In dollars. \\
\end{longtable}

\section{Lagged Variables}

Lagged variables are used as predictors in regressions to avoid simultaneity bias.

\begin{longtable}{p{3.5cm}p{8.5cm}}
\caption{Lagged Predictors}\label{tab:lags}\\
\toprule
\textbf{Variable} & \textbf{Definition \& Usage} \\
\midrule
\endfirsthead
\multicolumn{2}{c}{\textit{(continued from previous page)}}\\
\toprule
\textbf{Variable} & \textbf{Definition \& Usage} \\
\midrule
\endhead
\midrule
\multicolumn{2}{r}{\textit{(continued on next page)}}\\
\endfoot
\bottomrule
\endlastfoot

L\_equity\_ratio & One-period lagged equity ratio. Used in Scripts 08, 33, 34, 51. Predicts future failure. Range: 0.01-0.50. \\
L\_loan\_ratio & One-period lagged loan ratio. Used in same scripts. Captures asset risk one period prior. Range: 0.20-0.90. \\
L\_liquid\_ratio & One-period lagged liquid ratio. Used in same scripts. Prior period liquidity. Range: 0.05-0.60. \\
L\_log\_assets & One-period lagged log assets. Controls for prior bank size in regressions. Range: 5-20. \\
\end{longtable}

\section{Bank Run Variables}

\begin{longtable}{p{3.5cm}p{8.5cm}}
\caption{Deposit Run Indicators}\label{tab:runs}\\
\toprule
\textbf{Variable} & \textbf{Definition} \\
\midrule
\endfirsthead
\multicolumn{2}{c}{\textit{(continued from previous page)}}\\
\toprule
\textbf{Variable} & \textbf{Definition} \\
\midrule
\endhead
\midrule
\multicolumn{2}{r}{\textit{(continued on next page)}}\\
\endfoot
\bottomrule
\endlastfoot

run & Binary indicator for bank run. Equals 1 if deposit\_outflow $>$ 0.10 (10\% decline). Created in Script 06. \\
deposit\_outflow & Deposit decline rate. Formula: (last\_deposits - failure\_deposits) / last\_deposits. Range: -0.50 to +1.0. Positive = outflows (decline), negative = inflows (growth). \\
run\_is\_missing & Indicator for missing run data. Equals 1 if run information unavailable for that bank-year. \\
\end{longtable}

\section{Receivership Variables}

Variables related to bank failure outcomes and depositor recovery (historical era).

\begin{longtable}{p{3.5cm}p{8.5cm}}
\caption{Receivership Outcomes}\label{tab:receivership}\\
\toprule
\textbf{Variable} & \textbf{Definition} \\
\midrule
\endfirsthead
\multicolumn{2}{c}{\textit{(continued from previous page)}}\\
\toprule
\textbf{Variable} & \textbf{Definition} \\
\midrule
\endhead
\midrule
\multicolumn{2}{r}{\textit{(continued on next page)}}\\
\endfoot
\bottomrule
\endlastfoot

last\_call\_deposits & Deposits at last call report before failure. In nominal dollars. \\
last\_call\_assets & Assets at last call report before failure. In nominal dollars. \\
deposits\_at\_suspension & Deposits at suspension/failure date. From receivership records. \\
assets\_at\_suspension & Assets at suspension/failure date. From receivership records. \\
total\_assessed & Total assets assessed by receiver. Estimated liquidation value. \\
rho & Recovery rate. Calculated as total\_assessed / deposits\_at\_suspension. Range: 0-1. Mean: 0.0006 (0.06\%). \\
depositor\_loss & Depositor loss. Calculated as 1 - rho. Mean: 0.9994 (99.94\%). \\
receivership\_days & Days in receivership. From failure to final resolution. \\
receivership\_years & Years in receivership. receivership\_days / 365.25. \\
\end{longtable}

\section{Macro Variables}

Macroeconomic controls used in regressions.

\begin{longtable}{p{3.5cm}p{8.5cm}}
\caption{Macroeconomic Variables}\label{tab:macro}\\
\toprule
\textbf{Variable} & \textbf{Definition \& Source} \\
\midrule
\endfirsthead
\multicolumn{2}{c}{\textit{(continued from previous page)}}\\
\toprule
\textbf{Variable} & \textbf{Definition \& Source} \\
\midrule
\endhead
\midrule
\multicolumn{2}{r}{\textit{(continued on next page)}}\\
\endfoot
\bottomrule
\endlastfoot

rgdp\_pc & Real GDP per capita. From BEA (1947+) and Barro-Ursua (pre-1947). In constant dollars. \\
rgdp\_growth & Real GDP growth rate. Log difference of rgdp\_pc. \\
cpi & Consumer Price Index. From GFD and JST databases. Base year varies. \\
tbill\_rate & Treasury bill rate. Short-term interest rate from GFD. Percentage points. \\
bond\_yield & Long-term bond yield. From GFD. Percentage points. \\
crisis & Financial crisis indicator. From JST database. Binary: 1 = crisis year, 0 = normal. \\
\end{longtable}

\section{Variable Creation Scripts}

\subsection{Script Reference}

\begin{longtable}{p{2.5cm}p{9.5cm}}
\caption{Which Scripts Create Which Variables}\label{tab:script-reference}\\
\toprule
\textbf{Script(s)} & \textbf{Variables Created} \\
\midrule
\endfirsthead
\multicolumn{2}{c}{\textit{(continued from previous page)}}\\
\toprule
\textbf{Script(s)} & \textbf{Variables Created} \\
\midrule
\endhead
\midrule
\multicolumn{2}{r}{\textit{(continued on next page)}}\\
\endfoot
\bottomrule
\endlastfoot

04 & Historical balance sheet variables, age, failed\_bank, receivership\_date, quarters\_to\_failure \\
05 & Modern balance sheet variables, fail\_day, quarters\_to\_failure, quarter \\
06 & run, deposit\_outflow, receivership outcomes \\
07 & era\_group, combined panel \\
08 & Lagged variables (L\_equity\_ratio, L\_loan\_ratio, L\_liquid\_ratio, L\_log\_assets) for coefplots \\
32 & F1-F6\_failure indicators, growth, growth\_cat (cross-section analysis) \\
35 & F1/F3/F5\_failure, F1/F3\_failure\_run, growth, growth\_cat (main regression dataset) \\
51 & Uses lagged variables created in Script 08 for AUC analysis \\
\end{longtable}

\section{Key Variable Notes}

\subsection{Most Important Variables}

The following variables are central to the paper's analysis:

\begin{enumerate}
    \item \textbf{F3\_failure}: Primary outcome variable (failure within 3 years)
    \item \textbf{quarters\_to\_failure}: Core timing variable for all failure indicators
    \item \textbf{growth} \& \textbf{growth\_cat}: Primary finding - shrinking banks fail more
    \item \textbf{equity\_ratio}: Solvency measure, strong predictor of failure
    \item \textbf{liquid\_ratio}: Liquidity measure, important but weaker than equity
    \item \textbf{loan\_ratio}: Asset risk measure
    \item \textbf{run}: Bank run indicator, amplifies failure risk 3-4x
\end{enumerate}

\subsection{Era Differences}

Some variables differ by era:

\begin{itemize}
    \item \textbf{Historical (1863-1941)}: receivership\_date, surplus\_ratio, cash\_ratio
    \item \textbf{Modern (1959-2024)}: fail\_day, quarter variables, income\_ratio
    \item \textbf{Both eras}: Core balance sheet items, ratios, growth measures
\end{itemize}

\subsection{Missing Data Patterns}

\begin{itemize}
    \item \textbf{run}: Only available for subset of failing banks with receivership data
    \item \textbf{income\_ratio}: Only available post-1941 (modern reporting standards)
    \item \textbf{quarters\_to\_failure}: Only defined for failing banks (NA for survivors)
    \item \textbf{F[N]\_failure}: Only non-zero for failing banks within horizon
\end{itemize}

\section{Summary Statistics}

\subsection{Dataset Overview}

\begin{table}[h]
\centering
\caption{Combined Dataset Summary}
\begin{tabular}{lrr}
\toprule
\textbf{Statistic} & \textbf{Value} & \textbf{Notes} \\
\midrule
Total observations & 2,872,893 & Bank-year/quarters \\
Unique banks & ~30,000 & Historical + Modern \\
Failure rate (overall) & 7.7\% & Failed banks / all banks \\
Time span & 1863-2024 & 161 years \\
Historical obs & ~340,000 & 1863-1941, annual \\
Modern obs & ~2,500,000 & 1959-2024, quarterly \\
\bottomrule
\end{tabular}
\end{table}

\subsection{Key Variable Ranges}

\begin{table}[h]
\centering
\caption{Typical Values for Core Ratios}
\begin{tabular}{lrrr}
\toprule
\textbf{Variable} & \textbf{Mean} & \textbf{Median} & \textbf{Std Dev} \\
\midrule
equity\_ratio & 0.12 & 0.10 & 0.06 \\
loan\_ratio & 0.60 & 0.62 & 0.15 \\
liquid\_ratio & 0.25 & 0.22 & 0.12 \\
log\_assets & 12.5 & 12.3 & 2.1 \\
growth (3-year) & 0.15 & 0.12 & 0.35 \\
\bottomrule
\end{tabular}
\end{table}

\section{Conclusion}

This data dictionary provides comprehensive definitions for all variables in the Failing Banks R replication package. For questions about specific variables, see the technical documentation or examine the R scripts that create them.

\subsection{Additional Resources}

\begin{itemize}
    \item \textbf{Technical\_Documentation.pdf}: Detailed model specifications
    \item \textbf{Quick\_Start\_Guide.pdf}: How to run the replication
    \item \textbf{Source Scripts}: 1\_source\_code/ directory for variable creation logic
\end{itemize}

\end{document}

